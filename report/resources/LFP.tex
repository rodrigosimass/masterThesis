\documentclass{article}
\usepackage[utf8]{inputenc}

\title{Lex Fridman's Podcast summaries}
\author{R. Simas }
\date{September 2020}

\begin{document}

\maketitle
\section{episode 25}
\textbf{Jeff Hawkins}\newline
\begin{itemize}
\item "We wont be able to create fully intelligent machines until we understand how the human brain works"
\item The brain can be divided into two parts: the old part - lots of peaces; and the new part which is the neocortex (only mammals have it) and it occupies 75\% of the volume.
\item The old part handle basic behaviours, emotions and regulating the body.
\item The neocortex handles congnitive function and high-level perception. 
\item The neocortex is the same everywhere, it's uniform! In all species.
\item The neocortex grows by replicating itself - we believe the size of the cortex implies more intelligence by looking at different mammals.
\item Neuroscience literature has a lot of data but lacks a theoretical framework to unify it all- "It's a pre-paradime science"  as Khun would put it.
\item The brain works on \textbf{changing patterns}. HTM first principle was the temporal part of it. What CNNs do is: here is a picture classify it. What our eyes do is: move 5 times a second to see how an image changes over time. The second important principle for HTM was the memory: we learn a model of the world and we store it. The final principle is hierarchy: the neocortex has a hierarchical structure to it.
\item Every peace of empirical data on the brain is a constraint: HTM is a biologically constrained theory, not a biologically inspired one.
\item 
\end{itemize}

\bibliography{refs} 
\bibliographystyle{ieeetr}

\end{document}